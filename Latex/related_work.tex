\section{Related Work}

There are studies that analyze a CT feature and examine its importance in predicting adversarial outcomes like death or critical illness. \cite{BMD}analyzes the impact of Bone Mineral Density of L1 vertebra. It mentions the values ranges of ${<}$ 100 to be osteopric which we try to validate in our experiments. Similarly, \cite{FAT} analyzes the impact of various fat measures like Visceral Adipose tissue area and Subcutaneous Adipose tissue area on health and metabolism of patients. We use these observations to get patterns from our data using different ratios like Total adipose tissue area/Body area, VAT/SAT. 
Aortic calcification is a measure of calcium deposit in heart's blood vessels. As per common medical knowledge, it is a best indicator of Heart Attack. We analyze the work about this feature \cite{AoCa}, and confirm through our experiments. Liver HU indicates the attenuation of liver fat. The lower this value, the higher the risk of fatal outcomes like death. This is anlyzed in \cite{Liver HU}

There have been clinical researches \cite{common_methods_of_biological_age_estimation} \cite{60_new_50} \cite{bio_using_PCA} to exploit biological health markers to effectively calculate biological age. All these studies have incorporated different techniques such as PCA, Linear Regression, multiple linear regression (MLR), Klemera and Doubal’s method (KDM) etc. All these studies collected data on healthy individuals periodically to assess how biological health markers get affected as the chronological age grows. They have considered them a baseline and then use that to calculate biological age of test samples.

However, we don't have the same type and amount of data as those studies. So we cannot use them as it is. However, the inspiration is that biological health markers are effective at measuring biological health.